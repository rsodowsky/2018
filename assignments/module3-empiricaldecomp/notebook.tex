
% Default to the notebook output style

    


% Inherit from the specified cell style.




    
\documentclass[11pt]{article}

    
    
    \usepackage[T1]{fontenc}
    % Nicer default font (+ math font) than Computer Modern for most use cases
    \usepackage{mathpazo}

    % Basic figure setup, for now with no caption control since it's done
    % automatically by Pandoc (which extracts ![](path) syntax from Markdown).
    \usepackage{graphicx}
    % We will generate all images so they have a width \maxwidth. This means
    % that they will get their normal width if they fit onto the page, but
    % are scaled down if they would overflow the margins.
    \makeatletter
    \def\maxwidth{\ifdim\Gin@nat@width>\linewidth\linewidth
    \else\Gin@nat@width\fi}
    \makeatother
    \let\Oldincludegraphics\includegraphics
    % Set max figure width to be 80% of text width, for now hardcoded.
    \renewcommand{\includegraphics}[1]{\Oldincludegraphics[width=.8\maxwidth]{#1}}
    % Ensure that by default, figures have no caption (until we provide a
    % proper Figure object with a Caption API and a way to capture that
    % in the conversion process - todo).
    \usepackage{caption}
    \DeclareCaptionLabelFormat{nolabel}{}
    \captionsetup{labelformat=nolabel}

    \usepackage{adjustbox} % Used to constrain images to a maximum size 
    \usepackage{xcolor} % Allow colors to be defined
    \usepackage{enumerate} % Needed for markdown enumerations to work
    \usepackage{geometry} % Used to adjust the document margins
    \usepackage{amsmath} % Equations
    \usepackage{amssymb} % Equations
    \usepackage{textcomp} % defines textquotesingle
    % Hack from http://tex.stackexchange.com/a/47451/13684:
    \AtBeginDocument{%
        \def\PYZsq{\textquotesingle}% Upright quotes in Pygmentized code
    }
    \usepackage{upquote} % Upright quotes for verbatim code
    \usepackage{eurosym} % defines \euro
    \usepackage[mathletters]{ucs} % Extended unicode (utf-8) support
    \usepackage[utf8x]{inputenc} % Allow utf-8 characters in the tex document
    \usepackage{fancyvrb} % verbatim replacement that allows latex
    \usepackage{grffile} % extends the file name processing of package graphics 
                         % to support a larger range 
    % The hyperref package gives us a pdf with properly built
    % internal navigation ('pdf bookmarks' for the table of contents,
    % internal cross-reference links, web links for URLs, etc.)
    \usepackage{hyperref}
    \usepackage{longtable} % longtable support required by pandoc >1.10
    \usepackage{booktabs}  % table support for pandoc > 1.12.2
    \usepackage[inline]{enumitem} % IRkernel/repr support (it uses the enumerate* environment)
    \usepackage[normalem]{ulem} % ulem is needed to support strikethroughs (\sout)
                                % normalem makes italics be italics, not underlines
    

    
    
    % Colors for the hyperref package
    \definecolor{urlcolor}{rgb}{0,.145,.698}
    \definecolor{linkcolor}{rgb}{.71,0.21,0.01}
    \definecolor{citecolor}{rgb}{.12,.54,.11}

    % ANSI colors
    \definecolor{ansi-black}{HTML}{3E424D}
    \definecolor{ansi-black-intense}{HTML}{282C36}
    \definecolor{ansi-red}{HTML}{E75C58}
    \definecolor{ansi-red-intense}{HTML}{B22B31}
    \definecolor{ansi-green}{HTML}{00A250}
    \definecolor{ansi-green-intense}{HTML}{007427}
    \definecolor{ansi-yellow}{HTML}{DDB62B}
    \definecolor{ansi-yellow-intense}{HTML}{B27D12}
    \definecolor{ansi-blue}{HTML}{208FFB}
    \definecolor{ansi-blue-intense}{HTML}{0065CA}
    \definecolor{ansi-magenta}{HTML}{D160C4}
    \definecolor{ansi-magenta-intense}{HTML}{A03196}
    \definecolor{ansi-cyan}{HTML}{60C6C8}
    \definecolor{ansi-cyan-intense}{HTML}{258F8F}
    \definecolor{ansi-white}{HTML}{C5C1B4}
    \definecolor{ansi-white-intense}{HTML}{A1A6B2}

    % commands and environments needed by pandoc snippets
    % extracted from the output of `pandoc -s`
    \providecommand{\tightlist}{%
      \setlength{\itemsep}{0pt}\setlength{\parskip}{0pt}}
    \DefineVerbatimEnvironment{Highlighting}{Verbatim}{commandchars=\\\{\}}
    % Add ',fontsize=\small' for more characters per line
    \newenvironment{Shaded}{}{}
    \newcommand{\KeywordTok}[1]{\textcolor[rgb]{0.00,0.44,0.13}{\textbf{{#1}}}}
    \newcommand{\DataTypeTok}[1]{\textcolor[rgb]{0.56,0.13,0.00}{{#1}}}
    \newcommand{\DecValTok}[1]{\textcolor[rgb]{0.25,0.63,0.44}{{#1}}}
    \newcommand{\BaseNTok}[1]{\textcolor[rgb]{0.25,0.63,0.44}{{#1}}}
    \newcommand{\FloatTok}[1]{\textcolor[rgb]{0.25,0.63,0.44}{{#1}}}
    \newcommand{\CharTok}[1]{\textcolor[rgb]{0.25,0.44,0.63}{{#1}}}
    \newcommand{\StringTok}[1]{\textcolor[rgb]{0.25,0.44,0.63}{{#1}}}
    \newcommand{\CommentTok}[1]{\textcolor[rgb]{0.38,0.63,0.69}{\textit{{#1}}}}
    \newcommand{\OtherTok}[1]{\textcolor[rgb]{0.00,0.44,0.13}{{#1}}}
    \newcommand{\AlertTok}[1]{\textcolor[rgb]{1.00,0.00,0.00}{\textbf{{#1}}}}
    \newcommand{\FunctionTok}[1]{\textcolor[rgb]{0.02,0.16,0.49}{{#1}}}
    \newcommand{\RegionMarkerTok}[1]{{#1}}
    \newcommand{\ErrorTok}[1]{\textcolor[rgb]{1.00,0.00,0.00}{\textbf{{#1}}}}
    \newcommand{\NormalTok}[1]{{#1}}
    
    % Additional commands for more recent versions of Pandoc
    \newcommand{\ConstantTok}[1]{\textcolor[rgb]{0.53,0.00,0.00}{{#1}}}
    \newcommand{\SpecialCharTok}[1]{\textcolor[rgb]{0.25,0.44,0.63}{{#1}}}
    \newcommand{\VerbatimStringTok}[1]{\textcolor[rgb]{0.25,0.44,0.63}{{#1}}}
    \newcommand{\SpecialStringTok}[1]{\textcolor[rgb]{0.73,0.40,0.53}{{#1}}}
    \newcommand{\ImportTok}[1]{{#1}}
    \newcommand{\DocumentationTok}[1]{\textcolor[rgb]{0.73,0.13,0.13}{\textit{{#1}}}}
    \newcommand{\AnnotationTok}[1]{\textcolor[rgb]{0.38,0.63,0.69}{\textbf{\textit{{#1}}}}}
    \newcommand{\CommentVarTok}[1]{\textcolor[rgb]{0.38,0.63,0.69}{\textbf{\textit{{#1}}}}}
    \newcommand{\VariableTok}[1]{\textcolor[rgb]{0.10,0.09,0.49}{{#1}}}
    \newcommand{\ControlFlowTok}[1]{\textcolor[rgb]{0.00,0.44,0.13}{\textbf{{#1}}}}
    \newcommand{\OperatorTok}[1]{\textcolor[rgb]{0.40,0.40,0.40}{{#1}}}
    \newcommand{\BuiltInTok}[1]{{#1}}
    \newcommand{\ExtensionTok}[1]{{#1}}
    \newcommand{\PreprocessorTok}[1]{\textcolor[rgb]{0.74,0.48,0.00}{{#1}}}
    \newcommand{\AttributeTok}[1]{\textcolor[rgb]{0.49,0.56,0.16}{{#1}}}
    \newcommand{\InformationTok}[1]{\textcolor[rgb]{0.38,0.63,0.69}{\textbf{\textit{{#1}}}}}
    \newcommand{\WarningTok}[1]{\textcolor[rgb]{0.38,0.63,0.69}{\textbf{\textit{{#1}}}}}
    
    
    % Define a nice break command that doesn't care if a line doesn't already
    % exist.
    \def\br{\hspace*{\fill} \\* }
    % Math Jax compatability definitions
    \def\gt{>}
    \def\lt{<}
    % Document parameters
    \title{EOF\_vs\_fft-Zelinsky}
    
    
    

    % Pygments definitions
    
\makeatletter
\def\PY@reset{\let\PY@it=\relax \let\PY@bf=\relax%
    \let\PY@ul=\relax \let\PY@tc=\relax%
    \let\PY@bc=\relax \let\PY@ff=\relax}
\def\PY@tok#1{\csname PY@tok@#1\endcsname}
\def\PY@toks#1+{\ifx\relax#1\empty\else%
    \PY@tok{#1}\expandafter\PY@toks\fi}
\def\PY@do#1{\PY@bc{\PY@tc{\PY@ul{%
    \PY@it{\PY@bf{\PY@ff{#1}}}}}}}
\def\PY#1#2{\PY@reset\PY@toks#1+\relax+\PY@do{#2}}

\expandafter\def\csname PY@tok@w\endcsname{\def\PY@tc##1{\textcolor[rgb]{0.73,0.73,0.73}{##1}}}
\expandafter\def\csname PY@tok@c\endcsname{\let\PY@it=\textit\def\PY@tc##1{\textcolor[rgb]{0.25,0.50,0.50}{##1}}}
\expandafter\def\csname PY@tok@cp\endcsname{\def\PY@tc##1{\textcolor[rgb]{0.74,0.48,0.00}{##1}}}
\expandafter\def\csname PY@tok@k\endcsname{\let\PY@bf=\textbf\def\PY@tc##1{\textcolor[rgb]{0.00,0.50,0.00}{##1}}}
\expandafter\def\csname PY@tok@kp\endcsname{\def\PY@tc##1{\textcolor[rgb]{0.00,0.50,0.00}{##1}}}
\expandafter\def\csname PY@tok@kt\endcsname{\def\PY@tc##1{\textcolor[rgb]{0.69,0.00,0.25}{##1}}}
\expandafter\def\csname PY@tok@o\endcsname{\def\PY@tc##1{\textcolor[rgb]{0.40,0.40,0.40}{##1}}}
\expandafter\def\csname PY@tok@ow\endcsname{\let\PY@bf=\textbf\def\PY@tc##1{\textcolor[rgb]{0.67,0.13,1.00}{##1}}}
\expandafter\def\csname PY@tok@nb\endcsname{\def\PY@tc##1{\textcolor[rgb]{0.00,0.50,0.00}{##1}}}
\expandafter\def\csname PY@tok@nf\endcsname{\def\PY@tc##1{\textcolor[rgb]{0.00,0.00,1.00}{##1}}}
\expandafter\def\csname PY@tok@nc\endcsname{\let\PY@bf=\textbf\def\PY@tc##1{\textcolor[rgb]{0.00,0.00,1.00}{##1}}}
\expandafter\def\csname PY@tok@nn\endcsname{\let\PY@bf=\textbf\def\PY@tc##1{\textcolor[rgb]{0.00,0.00,1.00}{##1}}}
\expandafter\def\csname PY@tok@ne\endcsname{\let\PY@bf=\textbf\def\PY@tc##1{\textcolor[rgb]{0.82,0.25,0.23}{##1}}}
\expandafter\def\csname PY@tok@nv\endcsname{\def\PY@tc##1{\textcolor[rgb]{0.10,0.09,0.49}{##1}}}
\expandafter\def\csname PY@tok@no\endcsname{\def\PY@tc##1{\textcolor[rgb]{0.53,0.00,0.00}{##1}}}
\expandafter\def\csname PY@tok@nl\endcsname{\def\PY@tc##1{\textcolor[rgb]{0.63,0.63,0.00}{##1}}}
\expandafter\def\csname PY@tok@ni\endcsname{\let\PY@bf=\textbf\def\PY@tc##1{\textcolor[rgb]{0.60,0.60,0.60}{##1}}}
\expandafter\def\csname PY@tok@na\endcsname{\def\PY@tc##1{\textcolor[rgb]{0.49,0.56,0.16}{##1}}}
\expandafter\def\csname PY@tok@nt\endcsname{\let\PY@bf=\textbf\def\PY@tc##1{\textcolor[rgb]{0.00,0.50,0.00}{##1}}}
\expandafter\def\csname PY@tok@nd\endcsname{\def\PY@tc##1{\textcolor[rgb]{0.67,0.13,1.00}{##1}}}
\expandafter\def\csname PY@tok@s\endcsname{\def\PY@tc##1{\textcolor[rgb]{0.73,0.13,0.13}{##1}}}
\expandafter\def\csname PY@tok@sd\endcsname{\let\PY@it=\textit\def\PY@tc##1{\textcolor[rgb]{0.73,0.13,0.13}{##1}}}
\expandafter\def\csname PY@tok@si\endcsname{\let\PY@bf=\textbf\def\PY@tc##1{\textcolor[rgb]{0.73,0.40,0.53}{##1}}}
\expandafter\def\csname PY@tok@se\endcsname{\let\PY@bf=\textbf\def\PY@tc##1{\textcolor[rgb]{0.73,0.40,0.13}{##1}}}
\expandafter\def\csname PY@tok@sr\endcsname{\def\PY@tc##1{\textcolor[rgb]{0.73,0.40,0.53}{##1}}}
\expandafter\def\csname PY@tok@ss\endcsname{\def\PY@tc##1{\textcolor[rgb]{0.10,0.09,0.49}{##1}}}
\expandafter\def\csname PY@tok@sx\endcsname{\def\PY@tc##1{\textcolor[rgb]{0.00,0.50,0.00}{##1}}}
\expandafter\def\csname PY@tok@m\endcsname{\def\PY@tc##1{\textcolor[rgb]{0.40,0.40,0.40}{##1}}}
\expandafter\def\csname PY@tok@gh\endcsname{\let\PY@bf=\textbf\def\PY@tc##1{\textcolor[rgb]{0.00,0.00,0.50}{##1}}}
\expandafter\def\csname PY@tok@gu\endcsname{\let\PY@bf=\textbf\def\PY@tc##1{\textcolor[rgb]{0.50,0.00,0.50}{##1}}}
\expandafter\def\csname PY@tok@gd\endcsname{\def\PY@tc##1{\textcolor[rgb]{0.63,0.00,0.00}{##1}}}
\expandafter\def\csname PY@tok@gi\endcsname{\def\PY@tc##1{\textcolor[rgb]{0.00,0.63,0.00}{##1}}}
\expandafter\def\csname PY@tok@gr\endcsname{\def\PY@tc##1{\textcolor[rgb]{1.00,0.00,0.00}{##1}}}
\expandafter\def\csname PY@tok@ge\endcsname{\let\PY@it=\textit}
\expandafter\def\csname PY@tok@gs\endcsname{\let\PY@bf=\textbf}
\expandafter\def\csname PY@tok@gp\endcsname{\let\PY@bf=\textbf\def\PY@tc##1{\textcolor[rgb]{0.00,0.00,0.50}{##1}}}
\expandafter\def\csname PY@tok@go\endcsname{\def\PY@tc##1{\textcolor[rgb]{0.53,0.53,0.53}{##1}}}
\expandafter\def\csname PY@tok@gt\endcsname{\def\PY@tc##1{\textcolor[rgb]{0.00,0.27,0.87}{##1}}}
\expandafter\def\csname PY@tok@err\endcsname{\def\PY@bc##1{\setlength{\fboxsep}{0pt}\fcolorbox[rgb]{1.00,0.00,0.00}{1,1,1}{\strut ##1}}}
\expandafter\def\csname PY@tok@kc\endcsname{\let\PY@bf=\textbf\def\PY@tc##1{\textcolor[rgb]{0.00,0.50,0.00}{##1}}}
\expandafter\def\csname PY@tok@kd\endcsname{\let\PY@bf=\textbf\def\PY@tc##1{\textcolor[rgb]{0.00,0.50,0.00}{##1}}}
\expandafter\def\csname PY@tok@kn\endcsname{\let\PY@bf=\textbf\def\PY@tc##1{\textcolor[rgb]{0.00,0.50,0.00}{##1}}}
\expandafter\def\csname PY@tok@kr\endcsname{\let\PY@bf=\textbf\def\PY@tc##1{\textcolor[rgb]{0.00,0.50,0.00}{##1}}}
\expandafter\def\csname PY@tok@bp\endcsname{\def\PY@tc##1{\textcolor[rgb]{0.00,0.50,0.00}{##1}}}
\expandafter\def\csname PY@tok@fm\endcsname{\def\PY@tc##1{\textcolor[rgb]{0.00,0.00,1.00}{##1}}}
\expandafter\def\csname PY@tok@vc\endcsname{\def\PY@tc##1{\textcolor[rgb]{0.10,0.09,0.49}{##1}}}
\expandafter\def\csname PY@tok@vg\endcsname{\def\PY@tc##1{\textcolor[rgb]{0.10,0.09,0.49}{##1}}}
\expandafter\def\csname PY@tok@vi\endcsname{\def\PY@tc##1{\textcolor[rgb]{0.10,0.09,0.49}{##1}}}
\expandafter\def\csname PY@tok@vm\endcsname{\def\PY@tc##1{\textcolor[rgb]{0.10,0.09,0.49}{##1}}}
\expandafter\def\csname PY@tok@sa\endcsname{\def\PY@tc##1{\textcolor[rgb]{0.73,0.13,0.13}{##1}}}
\expandafter\def\csname PY@tok@sb\endcsname{\def\PY@tc##1{\textcolor[rgb]{0.73,0.13,0.13}{##1}}}
\expandafter\def\csname PY@tok@sc\endcsname{\def\PY@tc##1{\textcolor[rgb]{0.73,0.13,0.13}{##1}}}
\expandafter\def\csname PY@tok@dl\endcsname{\def\PY@tc##1{\textcolor[rgb]{0.73,0.13,0.13}{##1}}}
\expandafter\def\csname PY@tok@s2\endcsname{\def\PY@tc##1{\textcolor[rgb]{0.73,0.13,0.13}{##1}}}
\expandafter\def\csname PY@tok@sh\endcsname{\def\PY@tc##1{\textcolor[rgb]{0.73,0.13,0.13}{##1}}}
\expandafter\def\csname PY@tok@s1\endcsname{\def\PY@tc##1{\textcolor[rgb]{0.73,0.13,0.13}{##1}}}
\expandafter\def\csname PY@tok@mb\endcsname{\def\PY@tc##1{\textcolor[rgb]{0.40,0.40,0.40}{##1}}}
\expandafter\def\csname PY@tok@mf\endcsname{\def\PY@tc##1{\textcolor[rgb]{0.40,0.40,0.40}{##1}}}
\expandafter\def\csname PY@tok@mh\endcsname{\def\PY@tc##1{\textcolor[rgb]{0.40,0.40,0.40}{##1}}}
\expandafter\def\csname PY@tok@mi\endcsname{\def\PY@tc##1{\textcolor[rgb]{0.40,0.40,0.40}{##1}}}
\expandafter\def\csname PY@tok@il\endcsname{\def\PY@tc##1{\textcolor[rgb]{0.40,0.40,0.40}{##1}}}
\expandafter\def\csname PY@tok@mo\endcsname{\def\PY@tc##1{\textcolor[rgb]{0.40,0.40,0.40}{##1}}}
\expandafter\def\csname PY@tok@ch\endcsname{\let\PY@it=\textit\def\PY@tc##1{\textcolor[rgb]{0.25,0.50,0.50}{##1}}}
\expandafter\def\csname PY@tok@cm\endcsname{\let\PY@it=\textit\def\PY@tc##1{\textcolor[rgb]{0.25,0.50,0.50}{##1}}}
\expandafter\def\csname PY@tok@cpf\endcsname{\let\PY@it=\textit\def\PY@tc##1{\textcolor[rgb]{0.25,0.50,0.50}{##1}}}
\expandafter\def\csname PY@tok@c1\endcsname{\let\PY@it=\textit\def\PY@tc##1{\textcolor[rgb]{0.25,0.50,0.50}{##1}}}
\expandafter\def\csname PY@tok@cs\endcsname{\let\PY@it=\textit\def\PY@tc##1{\textcolor[rgb]{0.25,0.50,0.50}{##1}}}

\def\PYZbs{\char`\\}
\def\PYZus{\char`\_}
\def\PYZob{\char`\{}
\def\PYZcb{\char`\}}
\def\PYZca{\char`\^}
\def\PYZam{\char`\&}
\def\PYZlt{\char`\<}
\def\PYZgt{\char`\>}
\def\PYZsh{\char`\#}
\def\PYZpc{\char`\%}
\def\PYZdl{\char`\$}
\def\PYZhy{\char`\-}
\def\PYZsq{\char`\'}
\def\PYZdq{\char`\"}
\def\PYZti{\char`\~}
% for compatibility with earlier versions
\def\PYZat{@}
\def\PYZlb{[}
\def\PYZrb{]}
\makeatother


    % Exact colors from NB
    \definecolor{incolor}{rgb}{0.0, 0.0, 0.5}
    \definecolor{outcolor}{rgb}{0.545, 0.0, 0.0}



    
    % Prevent overflowing lines due to hard-to-break entities
    \sloppy 
    % Setup hyperref package
    \hypersetup{
      breaklinks=true,  % so long urls are correctly broken across lines
      colorlinks=true,
      urlcolor=urlcolor,
      linkcolor=linkcolor,
      citecolor=citecolor,
      }
    % Slightly bigger margins than the latex defaults
    
    \geometry{verbose,tmargin=1in,bmargin=1in,lmargin=1in,rmargin=1in}
    
    

    \begin{document}
    
    
    \maketitle
    
    

    
    \hypertarget{compress-fingerprint-data-using-pca}{%
\section{Compress Fingerprint data using
PCA}\label{compress-fingerprint-data-using-pca}}

\hypertarget{get-a-monochrome-fingerprint-image-from-google-image-search-and-read-it.}{%
\subsubsection{Get a monochrome Fingerprint image from google image
search and read
it.}\label{get-a-monochrome-fingerprint-image-from-google-image-search-and-read-it.}}

    \begin{Verbatim}[commandchars=\\\{\}]
{\color{incolor}In [{\color{incolor}9}]:} \PY{k+kn}{from} \PY{n+nn}{PIL} \PY{k}{import} \PY{n}{Image}
        \PY{k+kn}{import} \PY{n+nn}{urllib}
        \PY{k+kn}{from} \PY{n+nn}{io} \PY{k}{import} \PY{n}{StringIO}
        \PY{k+kn}{import} \PY{n+nn}{io}
        \PY{n}{img}\PY{o}{=}\PY{n}{Image}\PY{o}{.}\PY{n}{open}\PY{p}{(}\PY{n}{io}\PY{o}{.}\PY{n}{BytesIO}\PY{p}{(}\PY{n}{urllib}\PY{o}{.}\PY{n}{request}\PY{o}{.}\PY{n}{urlopen}\PY{p}{(}\PY{l+s+s1}{\PYZsq{}}\PY{l+s+s1}{https://people.sc.fsu.edu/\PYZti{}jburkardt/m\PYZus{}src/svd\PYZus{}fingerprint/fingerprint\PYZus{}01.png}\PY{l+s+s1}{\PYZsq{}}\PY{p}{)}\PY{o}{.}\PY{n}{read}\PY{p}{(}\PY{p}{)}\PY{p}{)}\PY{p}{)}
\end{Verbatim}


    \hypertarget{show-the-image-using-matplotlib.}{%
\subsubsection{Show the image using
matplotlib.}\label{show-the-image-using-matplotlib.}}

    \begin{Verbatim}[commandchars=\\\{\}]
{\color{incolor}In [{\color{incolor}10}]:} \PY{k+kn}{import} \PY{n+nn}{matplotlib}\PY{n+nn}{.}\PY{n+nn}{pyplot} \PY{k}{as} \PY{n+nn}{plt}
         \PY{o}{\PYZpc{}}\PY{k}{matplotlib} inline
         \PY{n}{plt}\PY{o}{.}\PY{n}{imshow}\PY{p}{(}\PY{n}{img}\PY{p}{)}
\end{Verbatim}


\begin{Verbatim}[commandchars=\\\{\}]
{\color{outcolor}Out[{\color{outcolor}10}]:} <matplotlib.image.AxesImage at 0x11dcd7c18>
\end{Verbatim}
            
    \begin{center}
    \adjustimage{max size={0.9\linewidth}{0.9\paperheight}}{output_3_1.png}
    \end{center}
    { \hspace*{\fill} \\}
    
    \hypertarget{convert-the-image-to-data-for-black0-and-white255-and-make-it-numpy-array}{%
\subsubsection{Convert the image to data for black(0) and white(255) and
make it numpy
array}\label{convert-the-image-to-data-for-black0-and-white255-and-make-it-numpy-array}}

    \begin{Verbatim}[commandchars=\\\{\}]
{\color{incolor}In [{\color{incolor}20}]:} \PY{n}{im}\PY{o}{=}\PY{n}{img}\PY{o}{.}\PY{n}{convert}\PY{p}{(}\PY{l+s+s1}{\PYZsq{}}\PY{l+s+s1}{L}\PY{l+s+s1}{\PYZsq{}}\PY{p}{)}
         \PY{k+kn}{import} \PY{n+nn}{numpy} \PY{k}{as} \PY{n+nn}{np}
         \PY{n}{imdata}\PY{o}{=}\PY{n}{np}\PY{o}{.}\PY{n}{asarray}\PY{p}{(}\PY{n}{im}\PY{p}{)}
         \PY{n+nb}{print}\PY{p}{(}\PY{l+s+s1}{\PYZsq{}}\PY{l+s+s1}{Image size is }\PY{l+s+s1}{\PYZsq{}}\PY{p}{)}
         \PY{n+nb}{print}\PY{p}{(}\PY{n}{imdata}\PY{o}{.}\PY{n}{shape}\PY{p}{)}
\end{Verbatim}


    \begin{Verbatim}[commandchars=\\\{\}]
Image size is 
(480, 400)

    \end{Verbatim}

    \hypertarget{check-the-array-by-reconverting-to-image.}{%
\subsubsection{Check the array by reconverting to
image.}\label{check-the-array-by-reconverting-to-image.}}

    \begin{Verbatim}[commandchars=\\\{\}]
{\color{incolor}In [{\color{incolor}58}]:} \PY{n}{plt}\PY{o}{.}\PY{n}{imshow}\PY{p}{(}\PY{n}{Image}\PY{o}{.}\PY{n}{fromarray}\PY{p}{(}\PY{n}{imdata}\PY{p}{)}\PY{o}{.}\PY{n}{convert}\PY{p}{(}\PY{l+s+s1}{\PYZsq{}}\PY{l+s+s1}{1}\PY{l+s+s1}{\PYZsq{}}\PY{p}{)}\PY{p}{)}
         \PY{n}{plt}\PY{o}{.}\PY{n}{imsave}\PY{p}{(}\PY{l+s+s1}{\PYZsq{}}\PY{l+s+s1}{img\PYZus{}full.png}\PY{l+s+s1}{\PYZsq{}}\PY{p}{,}\PY{n}{Image}\PY{o}{.}\PY{n}{fromarray}\PY{p}{(}\PY{n}{imdata}\PY{p}{)}\PY{o}{.}\PY{n}{convert}\PY{p}{(}\PY{l+s+s1}{\PYZsq{}}\PY{l+s+s1}{1}\PY{l+s+s1}{\PYZsq{}}\PY{p}{)}\PY{p}{)}
\end{Verbatim}


    \begin{center}
    \adjustimage{max size={0.9\linewidth}{0.9\paperheight}}{output_7_0.png}
    \end{center}
    { \hspace*{\fill} \\}
    
    \hypertarget{perform-svd-analysis-on-the-fingerprint-array.}{%
\subsubsection{Perform SVD analysis on the Fingerprint
array.}\label{perform-svd-analysis-on-the-fingerprint-array.}}

\hypertarget{question-how-many-pcs-do-you-expect}{%
\paragraph{Question: How many PCs do you
expect?}\label{question-how-many-pcs-do-you-expect}}

    \begin{Verbatim}[commandchars=\\\{\}]
{\color{incolor}In [{\color{incolor}22}]:} \PY{n}{u}\PY{p}{,}\PY{n}{s}\PY{p}{,}\PY{n}{v}\PY{o}{=}\PY{n}{np}\PY{o}{.}\PY{n}{linalg}\PY{o}{.}\PY{n}{svd}\PY{p}{(}\PY{n}{imdata}\PY{p}{)}
\end{Verbatim}


    \begin{Verbatim}[commandchars=\\\{\}]
{\color{incolor}In [{\color{incolor}24}]:} \PY{n+nb}{print}\PY{p}{(}\PY{n}{np}\PY{o}{.}\PY{n}{shape}\PY{p}{(}\PY{n}{u}\PY{p}{)}\PY{p}{)}
         \PY{n+nb}{print}\PY{p}{(}\PY{n}{np}\PY{o}{.}\PY{n}{shape}\PY{p}{(}\PY{n}{s}\PY{p}{)}\PY{p}{)}
         \PY{n+nb}{print}\PY{p}{(}\PY{n}{np}\PY{o}{.}\PY{n}{shape}\PY{p}{(}\PY{n}{v}\PY{p}{)}\PY{p}{)}
\end{Verbatim}


    \begin{Verbatim}[commandchars=\\\{\}]
(480, 480)
(400,)
(400, 400)

    \end{Verbatim}

    \hypertarget{question-how-many-pcs-are-good-for-approximation-the-original-data}{%
\paragraph{Question: How many PCs are good for approximation the
original
data?}\label{question-how-many-pcs-are-good-for-approximation-the-original-data}}

\hypertarget{plot-the-percentage-of-variance-explained-by-each-pc.}{%
\subsubsection{Plot the percentage of variance explained by each
PC.}\label{plot-the-percentage-of-variance-explained-by-each-pc.}}

    \begin{Verbatim}[commandchars=\\\{\}]
{\color{incolor}In [{\color{incolor}26}]:} \PY{n}{S}\PY{o}{=}\PY{n}{np}\PY{o}{.}\PY{n}{diag}\PY{p}{(}\PY{n}{s}\PY{p}{)}        \PY{c+c1}{\PYZsh{} s is 1d, this is how numpy reports s, np.diag makes it a size(s)*size(s) matrix with s as diagnol}
         \PY{n}{var\PYZus{}or\PYZus{}eig}\PY{o}{=}\PY{n}{S}\PY{o}{.}\PY{n}{T}\PY{o}{.}\PY{n}{dot}\PY{p}{(}\PY{n}{S}\PY{p}{)}  \PY{c+c1}{\PYZsh{} this is nothing but eigen values of covariance matrix}
         \PY{n}{var1d}\PY{o}{=}\PY{n}{np}\PY{o}{.}\PY{n}{diag}\PY{p}{(}\PY{n}{var\PYZus{}or\PYZus{}eig}\PY{p}{)}     \PY{c+c1}{\PYZsh{} to make it 1d from diagnol matrix, all other elements are zeors}
         \PY{n}{var1dpct}\PY{o}{=}\PY{n}{var1d}\PY{o}{*}\PY{l+m+mf}{100.0}\PY{o}{/}\PY{n+nb}{sum}\PY{p}{(}\PY{n}{var1d}\PY{p}{)}  \PY{c+c1}{\PYZsh{}compute variance as percent}
\end{Verbatim}


    \begin{Verbatim}[commandchars=\\\{\}]
{\color{incolor}In [{\color{incolor}27}]:} \PY{n}{plt}\PY{o}{.}\PY{n}{plot}\PY{p}{(}\PY{n}{var1dpct}\PY{p}{,}\PY{n}{label}\PY{o}{=}\PY{l+s+s2}{\PYZdq{}}\PY{l+s+s2}{Variance as Percent}\PY{l+s+s2}{\PYZdq{}}\PY{p}{)}
         \PY{n}{plt}\PY{o}{.}\PY{n}{plot}\PY{p}{(}\PY{n}{var1dpct}\PY{o}{.}\PY{n}{cumsum}\PY{p}{(}\PY{p}{)}\PY{p}{,}\PY{n}{label}\PY{o}{=}\PY{l+s+s2}{\PYZdq{}}\PY{l+s+s2}{Cummulative Variance}\PY{l+s+s2}{\PYZdq{}}\PY{p}{)}
         \PY{n}{plt}\PY{o}{.}\PY{n}{legend}\PY{p}{(}\PY{p}{)}
\end{Verbatim}


\begin{Verbatim}[commandchars=\\\{\}]
{\color{outcolor}Out[{\color{outcolor}27}]:} <matplotlib.legend.Legend at 0x11ef7d860>
\end{Verbatim}
            
    \begin{center}
    \adjustimage{max size={0.9\linewidth}{0.9\paperheight}}{output_13_1.png}
    \end{center}
    { \hspace*{\fill} \\}
    
    \hypertarget{looks-like-first-few-pcs-contain-most-variance-of-the-data.}{%
\paragraph{Looks like first few PCs contain most variance of the
data.}\label{looks-like-first-few-pcs-contain-most-variance-of-the-data.}}

\hypertarget{reconstruct-data-from-pcs}{%
\subsection{Reconstruct data from PCs}\label{reconstruct-data-from-pcs}}

\hypertarget{using-2-pcs-to-reconstruct-the-data}{%
\subsubsection{1. Using 2 PCs to reconstruct the
data}\label{using-2-pcs-to-reconstruct-the-data}}

    \begin{Verbatim}[commandchars=\\\{\}]
{\color{incolor}In [{\color{incolor}28}]:} \PY{n}{img\PYZus{}data\PYZus{}2PC}\PY{o}{=}\PY{n}{u}\PY{p}{[}\PY{p}{:}\PY{p}{,}\PY{l+m+mi}{0}\PY{p}{:}\PY{l+m+mi}{2}\PY{p}{]}\PY{o}{.}\PY{n}{dot}\PY{p}{(}\PY{n}{S}\PY{p}{[}\PY{l+m+mi}{0}\PY{p}{:}\PY{l+m+mi}{2}\PY{p}{,}\PY{l+m+mi}{0}\PY{p}{:}\PY{l+m+mi}{2}\PY{p}{]}\PY{p}{)}\PY{o}{.}\PY{n}{dot}\PY{p}{(}\PY{n}{v}\PY{p}{[}\PY{l+m+mi}{0}\PY{p}{:}\PY{l+m+mi}{2}\PY{p}{,}\PY{p}{:}\PY{p}{]}\PY{p}{)}
\end{Verbatim}


    \begin{Verbatim}[commandchars=\\\{\}]
{\color{incolor}In [{\color{incolor}37}]:} \PY{n}{plt}\PY{o}{.}\PY{n}{imshow}\PY{p}{(}\PY{n}{Image}\PY{o}{.}\PY{n}{fromarray}\PY{p}{(}\PY{n}{img\PYZus{}data\PYZus{}2PC}\PY{p}{)}\PY{o}{.}\PY{n}{convert}\PY{p}{(}\PY{l+s+s1}{\PYZsq{}}\PY{l+s+s1}{1}\PY{l+s+s1}{\PYZsq{}}\PY{p}{)}\PY{p}{)}
         \PY{n+nb}{print}\PY{p}{(}\PY{l+s+s1}{\PYZsq{}}\PY{l+s+s1}{The percent of variance from the original image}\PY{l+s+s1}{\PYZsq{}}\PY{p}{)}
         \PY{n+nb}{print}\PY{p}{(}\PY{n}{np}\PY{o}{.}\PY{n}{sum}\PY{p}{(}\PY{n}{var1dpct}\PY{p}{[}\PY{l+m+mi}{0}\PY{p}{:}\PY{l+m+mi}{2}\PY{p}{]}\PY{p}{)}\PY{p}{)}
\end{Verbatim}


    \begin{Verbatim}[commandchars=\\\{\}]
The percent of variance from the original image
73.63662794146275

    \end{Verbatim}

    \begin{center}
    \adjustimage{max size={0.9\linewidth}{0.9\paperheight}}{output_16_1.png}
    \end{center}
    { \hspace*{\fill} \\}
    
    \hypertarget{using-10-pcs-to-reconstruct-the-data.}{%
\subsubsection{2. Using 10 PCs to reconstruct the
data.}\label{using-10-pcs-to-reconstruct-the-data.}}

    \begin{Verbatim}[commandchars=\\\{\}]
{\color{incolor}In [{\color{incolor}55}]:} \PY{n}{img\PYZus{}data\PYZus{}10PC}\PY{o}{=}\PY{n}{u}\PY{p}{[}\PY{p}{:}\PY{p}{,}\PY{l+m+mi}{0}\PY{p}{:}\PY{l+m+mi}{10}\PY{p}{]}\PY{o}{.}\PY{n}{dot}\PY{p}{(}\PY{n}{S}\PY{p}{[}\PY{l+m+mi}{0}\PY{p}{:}\PY{l+m+mi}{10}\PY{p}{,}\PY{l+m+mi}{0}\PY{p}{:}\PY{l+m+mi}{10}\PY{p}{]}\PY{p}{)}\PY{o}{.}\PY{n}{dot}\PY{p}{(}\PY{n}{v}\PY{p}{[}\PY{l+m+mi}{0}\PY{p}{:}\PY{l+m+mi}{10}\PY{p}{,}\PY{p}{:}\PY{p}{]}\PY{p}{)}
\end{Verbatim}


    \begin{Verbatim}[commandchars=\\\{\}]
192000

    \end{Verbatim}

    \begin{Verbatim}[commandchars=\\\{\}]
{\color{incolor}In [{\color{incolor}60}]:} \PY{n}{plt}\PY{o}{.}\PY{n}{imshow}\PY{p}{(}\PY{n}{Image}\PY{o}{.}\PY{n}{fromarray}\PY{p}{(}\PY{n}{img\PYZus{}data\PYZus{}10PC}\PY{p}{)}\PY{o}{.}\PY{n}{convert}\PY{p}{(}\PY{l+s+s1}{\PYZsq{}}\PY{l+s+s1}{1}\PY{l+s+s1}{\PYZsq{}}\PY{p}{)}\PY{p}{)}
         \PY{n+nb}{print}\PY{p}{(}\PY{l+s+s1}{\PYZsq{}}\PY{l+s+s1}{The percent of variance from the original image}\PY{l+s+s1}{\PYZsq{}}\PY{p}{)}
         \PY{n+nb}{print}\PY{p}{(}\PY{n}{np}\PY{o}{.}\PY{n}{sum}\PY{p}{(}\PY{n}{var1dpct}\PY{p}{[}\PY{l+m+mi}{0}\PY{p}{:}\PY{l+m+mi}{10}\PY{p}{]}\PY{p}{)}\PY{p}{)}
         \PY{n}{plt}\PY{o}{.}\PY{n}{imsave}\PY{p}{(}\PY{l+s+s1}{\PYZsq{}}\PY{l+s+s1}{img\PYZus{}reduced.png}\PY{l+s+s1}{\PYZsq{}}\PY{p}{,}\PY{n}{Image}\PY{o}{.}\PY{n}{fromarray}\PY{p}{(}\PY{n}{img\PYZus{}data\PYZus{}10PC}\PY{p}{)}\PY{o}{.}\PY{n}{convert}\PY{p}{(}\PY{l+s+s1}{\PYZsq{}}\PY{l+s+s1}{1}\PY{l+s+s1}{\PYZsq{}}\PY{p}{)}\PY{p}{)}
\end{Verbatim}


    \begin{Verbatim}[commandchars=\\\{\}]
The percent of variance from the original image
82.4304580983708

    \end{Verbatim}

    \begin{center}
    \adjustimage{max size={0.9\linewidth}{0.9\paperheight}}{output_19_1.png}
    \end{center}
    { \hspace*{\fill} \\}
    
    \begin{Verbatim}[commandchars=\\\{\}]
{\color{incolor}In [{\color{incolor}130}]:} \PY{n}{img\PYZus{}data\PYZus{}20PC}\PY{o}{=}\PY{n}{u}\PY{p}{[}\PY{p}{:}\PY{p}{,}\PY{l+m+mi}{0}\PY{p}{:}\PY{l+m+mi}{25}\PY{p}{]}\PY{o}{.}\PY{n}{dot}\PY{p}{(}\PY{n}{S}\PY{p}{[}\PY{l+m+mi}{0}\PY{p}{:}\PY{l+m+mi}{25}\PY{p}{,}\PY{l+m+mi}{0}\PY{p}{:}\PY{l+m+mi}{25}\PY{p}{]}\PY{p}{)}\PY{o}{.}\PY{n}{dot}\PY{p}{(}\PY{n}{v}\PY{p}{[}\PY{l+m+mi}{0}\PY{p}{:}\PY{l+m+mi}{25}\PY{p}{,}\PY{p}{:}\PY{p}{]}\PY{p}{)}
          \PY{n}{plt}\PY{o}{.}\PY{n}{imshow}\PY{p}{(}\PY{n}{Image}\PY{o}{.}\PY{n}{fromarray}\PY{p}{(}\PY{n}{img\PYZus{}data\PYZus{}20PC}\PY{p}{)}\PY{o}{.}\PY{n}{convert}\PY{p}{(}\PY{l+s+s1}{\PYZsq{}}\PY{l+s+s1}{1}\PY{l+s+s1}{\PYZsq{}}\PY{p}{)}\PY{p}{)}
          \PY{n+nb}{print}\PY{p}{(}\PY{l+s+s1}{\PYZsq{}}\PY{l+s+s1}{The percent of variance from the original image}\PY{l+s+s1}{\PYZsq{}}\PY{p}{)}
          \PY{n+nb}{print}\PY{p}{(}\PY{n}{np}\PY{o}{.}\PY{n}{sum}\PY{p}{(}\PY{n}{var1dpct}\PY{p}{[}\PY{l+m+mi}{0}\PY{p}{:}\PY{l+m+mi}{25}\PY{p}{]}\PY{p}{)}\PY{p}{)}
          \PY{n}{plt}\PY{o}{.}\PY{n}{imsave}\PY{p}{(}\PY{l+s+s1}{\PYZsq{}}\PY{l+s+s1}{img\PYZus{}reduced.png}\PY{l+s+s1}{\PYZsq{}}\PY{p}{,}\PY{n}{Image}\PY{o}{.}\PY{n}{fromarray}\PY{p}{(}\PY{n}{img\PYZus{}data\PYZus{}20PC}\PY{p}{)}\PY{o}{.}\PY{n}{convert}\PY{p}{(}\PY{l+s+s1}{\PYZsq{}}\PY{l+s+s1}{1}\PY{l+s+s1}{\PYZsq{}}\PY{p}{)}\PY{p}{)}
\end{Verbatim}


    \begin{Verbatim}[commandchars=\\\{\}]
The percent of variance from the original image
90.13294711894079

    \end{Verbatim}

    \begin{center}
    \adjustimage{max size={0.9\linewidth}{0.9\paperheight}}{output_20_1.png}
    \end{center}
    { \hspace*{\fill} \\}
    
    \hypertarget{now-try-filtering-the-image-with-fft}{%
\subsection{Now Try filtering the image with
fft}\label{now-try-filtering-the-image-with-fft}}

    \begin{Verbatim}[commandchars=\\\{\}]
{\color{incolor}In [{\color{incolor}93}]:} \PY{k+kn}{from} \PY{n+nn}{scipy} \PY{k}{import} \PY{n}{fftpack}
         \PY{n}{im\PYZus{}fft} \PY{o}{=} \PY{n}{fftpack}\PY{o}{.}\PY{n}{fft2}\PY{p}{(}\PY{n}{img}\PY{p}{)}
         
         \PY{k}{def} \PY{n+nf}{plot\PYZus{}spectrum}\PY{p}{(}\PY{n}{im\PYZus{}fft}\PY{p}{)}\PY{p}{:}
             \PY{k+kn}{from} \PY{n+nn}{matplotlib}\PY{n+nn}{.}\PY{n+nn}{colors} \PY{k}{import} \PY{n}{LogNorm}
             \PY{c+c1}{\PYZsh{} A logarithmic colormap}
             \PY{n}{plt}\PY{o}{.}\PY{n}{imshow}\PY{p}{(}\PY{n}{np}\PY{o}{.}\PY{n}{abs}\PY{p}{(}\PY{n}{im\PYZus{}fft}\PY{p}{)}\PY{p}{,} \PY{n}{norm}\PY{o}{=}\PY{n}{LogNorm}\PY{p}{(}\PY{n}{vmin}\PY{o}{=}\PY{l+m+mi}{5}\PY{p}{)}\PY{p}{)}
             \PY{n}{plt}\PY{o}{.}\PY{n}{colorbar}\PY{p}{(}\PY{p}{)}
         
         \PY{n}{plt}\PY{o}{.}\PY{n}{figure}\PY{p}{(}\PY{p}{)}
         \PY{n}{plot\PYZus{}spectrum}\PY{p}{(}\PY{n}{im\PYZus{}fft}\PY{p}{)}
         \PY{n}{plt}\PY{o}{.}\PY{n}{title}\PY{p}{(}\PY{l+s+s1}{\PYZsq{}}\PY{l+s+s1}{Fourier transform}\PY{l+s+s1}{\PYZsq{}}\PY{p}{)}
\end{Verbatim}


\begin{Verbatim}[commandchars=\\\{\}]
{\color{outcolor}Out[{\color{outcolor}93}]:} Text(0.5,1,'Fourier transform')
\end{Verbatim}
            
    \begin{center}
    \adjustimage{max size={0.9\linewidth}{0.9\paperheight}}{output_22_1.png}
    \end{center}
    { \hspace*{\fill} \\}
    
    \begin{Verbatim}[commandchars=\\\{\}]
{\color{incolor}In [{\color{incolor}94}]:} \PY{c+c1}{\PYZsh{} Define the fraction of coefficients (in each direction) we keep}
         \PY{n}{keep\PYZus{}fraction} \PY{o}{=} \PY{l+m+mf}{0.1}
         
         \PY{c+c1}{\PYZsh{} Call ff a copy of the original transform. Numpy arrays have a copy}
         \PY{c+c1}{\PYZsh{} method for this purpose.}
         \PY{n}{im\PYZus{}fft2} \PY{o}{=} \PY{n}{im\PYZus{}fft}\PY{o}{.}\PY{n}{copy}\PY{p}{(}\PY{p}{)}
         
         \PY{c+c1}{\PYZsh{} Set r and c to be the number of rows and columns of the array.}
         \PY{n}{r}\PY{p}{,} \PY{n}{c} \PY{o}{=} \PY{n}{im\PYZus{}fft2}\PY{o}{.}\PY{n}{shape}
         
         \PY{c+c1}{\PYZsh{} Set to zero all rows with indices between r*keep\PYZus{}fraction and}
         \PY{c+c1}{\PYZsh{} r*(1\PYZhy{}keep\PYZus{}fraction):}
         \PY{n}{im\PYZus{}fft2}\PY{p}{[}\PY{n+nb}{int}\PY{p}{(}\PY{n}{r}\PY{o}{*}\PY{n}{keep\PYZus{}fraction}\PY{p}{)}\PY{p}{:}\PY{n+nb}{int}\PY{p}{(}\PY{n}{r}\PY{o}{*}\PY{p}{(}\PY{l+m+mi}{1}\PY{o}{\PYZhy{}}\PY{n}{keep\PYZus{}fraction}\PY{p}{)}\PY{p}{)}\PY{p}{]} \PY{o}{=} \PY{l+m+mi}{0}
         
         \PY{c+c1}{\PYZsh{} Similarly with the columns:}
         \PY{n}{im\PYZus{}fft2}\PY{p}{[}\PY{p}{:}\PY{p}{,} \PY{n+nb}{int}\PY{p}{(}\PY{n}{c}\PY{o}{*}\PY{n}{keep\PYZus{}fraction}\PY{p}{)}\PY{p}{:}\PY{n+nb}{int}\PY{p}{(}\PY{n}{c}\PY{o}{*}\PY{p}{(}\PY{l+m+mi}{1}\PY{o}{\PYZhy{}}\PY{n}{keep\PYZus{}fraction}\PY{p}{)}\PY{p}{)}\PY{p}{]} \PY{o}{=} \PY{l+m+mi}{0}
         
         \PY{n}{plt}\PY{o}{.}\PY{n}{figure}\PY{p}{(}\PY{p}{)}
         \PY{n}{plot\PYZus{}spectrum}\PY{p}{(}\PY{n}{im\PYZus{}fft2}\PY{p}{)}
         \PY{n}{plt}\PY{o}{.}\PY{n}{title}\PY{p}{(}\PY{l+s+s1}{\PYZsq{}}\PY{l+s+s1}{Filtered Spectrum}\PY{l+s+s1}{\PYZsq{}}\PY{p}{)}
\end{Verbatim}


\begin{Verbatim}[commandchars=\\\{\}]
{\color{outcolor}Out[{\color{outcolor}94}]:} Text(0.5,1,'Filtered Spectrum')
\end{Verbatim}
            
    \begin{center}
    \adjustimage{max size={0.9\linewidth}{0.9\paperheight}}{output_23_1.png}
    \end{center}
    { \hspace*{\fill} \\}
    
    \begin{Verbatim}[commandchars=\\\{\}]
{\color{incolor}In [{\color{incolor}95}]:} \PY{c+c1}{\PYZsh{} Reconstruct the image from the filtered spectrum, keep only the}
         \PY{c+c1}{\PYZsh{} real part for display.}
         \PY{n}{im\PYZus{}new} \PY{o}{=} \PY{n}{fftpack}\PY{o}{.}\PY{n}{ifft2}\PY{p}{(}\PY{n}{im\PYZus{}fft2}\PY{p}{)}\PY{o}{.}\PY{n}{real}
         
         \PY{n}{plt}\PY{o}{.}\PY{n}{figure}\PY{p}{(}\PY{p}{)}
         \PY{n}{plt}\PY{o}{.}\PY{n}{imshow}\PY{p}{(}\PY{n}{im\PYZus{}new}\PY{p}{,} \PY{n}{plt}\PY{o}{.}\PY{n}{cm}\PY{o}{.}\PY{n}{gray}\PY{p}{)}
         \PY{n}{plt}\PY{o}{.}\PY{n}{title}\PY{p}{(}\PY{l+s+s1}{\PYZsq{}}\PY{l+s+s1}{Reconstructed Image}\PY{l+s+s1}{\PYZsq{}}\PY{p}{)}
\end{Verbatim}


\begin{Verbatim}[commandchars=\\\{\}]
{\color{outcolor}Out[{\color{outcolor}95}]:} Text(0.5,1,'Reconstructed Image')
\end{Verbatim}
            
    \begin{center}
    \adjustimage{max size={0.9\linewidth}{0.9\paperheight}}{output_24_1.png}
    \end{center}
    { \hspace*{\fill} \\}
    
    \begin{Verbatim}[commandchars=\\\{\}]
{\color{incolor}In [{\color{incolor}112}]:} \PY{c+c1}{\PYZsh{} Define the fraction of coefficients (in each direction) we keep}
          \PY{n}{keep\PYZus{}fraction} \PY{o}{=} \PY{l+m+mf}{0.3}
          
          \PY{c+c1}{\PYZsh{} Call ff a copy of the original transform. Numpy arrays have a copy}
          \PY{c+c1}{\PYZsh{} method for this purpose.}
          \PY{n}{im\PYZus{}fft3} \PY{o}{=} \PY{n}{im\PYZus{}fft}\PY{o}{.}\PY{n}{copy}\PY{p}{(}\PY{p}{)}
          
          \PY{c+c1}{\PYZsh{} Set r and c to be the number of rows and columns of the array.}
          \PY{n}{r}\PY{p}{,} \PY{n}{c} \PY{o}{=} \PY{n}{im\PYZus{}fft2}\PY{o}{.}\PY{n}{shape}
          
          \PY{c+c1}{\PYZsh{} Set to zero all rows with indices between 0 :r*keep\PYZus{}fraction and}
          \PY{c+c1}{\PYZsh{} r*(1\PYZhy{}keep\PYZus{}fraction):end }
          \PY{n}{im\PYZus{}fft3}\PY{p}{[}\PY{l+m+mi}{0}\PY{p}{:}\PY{n+nb}{int}\PY{p}{(}\PY{n}{r}\PY{o}{*}\PY{n}{keep\PYZus{}fraction}\PY{p}{)}\PY{p}{]} \PY{o}{=} \PY{l+m+mi}{0}
          \PY{n}{im\PYZus{}fft3}\PY{p}{[}\PY{n+nb}{int}\PY{p}{(}\PY{n}{r}\PY{o}{*}\PY{p}{(}\PY{l+m+mi}{1}\PY{o}{\PYZhy{}}\PY{n}{keep\PYZus{}fraction}\PY{p}{)}\PY{p}{)}\PY{p}{:}\PY{o}{\PYZhy{}}\PY{l+m+mi}{1}\PY{p}{]} \PY{o}{=} \PY{l+m+mi}{0}
          
          \PY{c+c1}{\PYZsh{} Similarly with the columns:}
          \PY{n}{im\PYZus{}fft3}\PY{p}{[}\PY{p}{:}\PY{p}{,} \PY{l+m+mi}{0}\PY{p}{:}\PY{n+nb}{int}\PY{p}{(}\PY{n}{c}\PY{o}{*}\PY{n}{keep\PYZus{}fraction}\PY{p}{)}\PY{p}{]} \PY{o}{=} \PY{l+m+mi}{0}
          \PY{n}{im\PYZus{}fft3}\PY{p}{[}\PY{p}{:}\PY{p}{,} \PY{n+nb}{int}\PY{p}{(}\PY{n}{c}\PY{o}{*}\PY{p}{(}\PY{l+m+mi}{1}\PY{o}{\PYZhy{}}\PY{n}{keep\PYZus{}fraction}\PY{p}{)}\PY{p}{)}\PY{p}{:}\PY{o}{\PYZhy{}}\PY{l+m+mi}{1}\PY{p}{]} \PY{o}{=} \PY{l+m+mi}{0}
          
          \PY{n}{plt}\PY{o}{.}\PY{n}{figure}\PY{p}{(}\PY{p}{)}
          \PY{n}{plot\PYZus{}spectrum}\PY{p}{(}\PY{n}{im\PYZus{}fft3}\PY{p}{)}
          \PY{n}{plt}\PY{o}{.}\PY{n}{title}\PY{p}{(}\PY{l+s+s1}{\PYZsq{}}\PY{l+s+s1}{Filtered Spectrum}\PY{l+s+s1}{\PYZsq{}}\PY{p}{)}
\end{Verbatim}


\begin{Verbatim}[commandchars=\\\{\}]
{\color{outcolor}Out[{\color{outcolor}112}]:} Text(0.5,1,'Filtered Spectrum')
\end{Verbatim}
            
    \begin{center}
    \adjustimage{max size={0.9\linewidth}{0.9\paperheight}}{output_25_1.png}
    \end{center}
    { \hspace*{\fill} \\}
    
    \begin{Verbatim}[commandchars=\\\{\}]
{\color{incolor}In [{\color{incolor}113}]:} \PY{c+c1}{\PYZsh{} Reconstruct the image from the filtered spectrum, keep only the}
          \PY{c+c1}{\PYZsh{} real part for display.}
          \PY{n}{im\PYZus{}new} \PY{o}{=} \PY{n}{fftpack}\PY{o}{.}\PY{n}{ifft2}\PY{p}{(}\PY{n}{im\PYZus{}fft3}\PY{p}{)}\PY{o}{.}\PY{n}{real}
          
          \PY{n}{plt}\PY{o}{.}\PY{n}{figure}\PY{p}{(}\PY{p}{)}
          \PY{n}{plt}\PY{o}{.}\PY{n}{imshow}\PY{p}{(}\PY{n}{im\PYZus{}new}\PY{p}{,} \PY{n}{plt}\PY{o}{.}\PY{n}{cm}\PY{o}{.}\PY{n}{gray}\PY{p}{)}
          \PY{n}{plt}\PY{o}{.}\PY{n}{title}\PY{p}{(}\PY{l+s+s1}{\PYZsq{}}\PY{l+s+s1}{Reconstructed Image}\PY{l+s+s1}{\PYZsq{}}\PY{p}{)}
\end{Verbatim}


\begin{Verbatim}[commandchars=\\\{\}]
{\color{outcolor}Out[{\color{outcolor}113}]:} Text(0.5,1,'Reconstructed Image')
\end{Verbatim}
            
    \begin{center}
    \adjustimage{max size={0.9\linewidth}{0.9\paperheight}}{output_26_1.png}
    \end{center}
    { \hspace*{\fill} \\}
    
    \begin{Verbatim}[commandchars=\\\{\}]
{\color{incolor}In [{\color{incolor}119}]:} \PY{c+c1}{\PYZsh{} Define the fraction of coefficients (in each direction) we keep}
          \PY{n}{keep\PYZus{}fraction} \PY{o}{=} \PY{l+m+mf}{0.5}
          
          \PY{c+c1}{\PYZsh{} Call ff a copy of the original transform. Numpy arrays have a copy}
          \PY{c+c1}{\PYZsh{} method for this purpose.}
          \PY{n}{im\PYZus{}fft4} \PY{o}{=} \PY{n}{im\PYZus{}fft}\PY{o}{.}\PY{n}{copy}\PY{p}{(}\PY{p}{)}
          
          \PY{c+c1}{\PYZsh{} Set r and c to be the number of rows and columns of the array.}
          \PY{n}{r}\PY{p}{,} \PY{n}{c} \PY{o}{=} \PY{n}{im\PYZus{}fft4}\PY{o}{.}\PY{n}{shape}
          
          \PY{c+c1}{\PYZsh{} Set to zero all rows with indices between 0 :r*keep\PYZus{}fraction and}
          \PY{c+c1}{\PYZsh{} r*(1\PYZhy{}keep\PYZus{}fraction):end }
          \PY{n}{im\PYZus{}fft4}\PY{p}{[}\PY{l+m+mi}{0}\PY{p}{:}\PY{n+nb}{int}\PY{p}{(}\PY{n}{r}\PY{o}{*}\PY{n}{keep\PYZus{}fraction}\PY{p}{)}\PY{p}{]} \PY{o}{=} \PY{l+m+mi}{0}
          
          \PY{c+c1}{\PYZsh{} Similarly with the columns:}
          \PY{c+c1}{\PYZsh{}im\PYZus{}fft4[:, 0:int(c*keep\PYZus{}fraction)] = 0}
          
          \PY{n}{plt}\PY{o}{.}\PY{n}{figure}\PY{p}{(}\PY{p}{)}
          \PY{n}{plot\PYZus{}spectrum}\PY{p}{(}\PY{n}{im\PYZus{}fft4}\PY{p}{)}
          \PY{n}{plt}\PY{o}{.}\PY{n}{title}\PY{p}{(}\PY{l+s+s1}{\PYZsq{}}\PY{l+s+s1}{Filtered Spectrum}\PY{l+s+s1}{\PYZsq{}}\PY{p}{)}
\end{Verbatim}


\begin{Verbatim}[commandchars=\\\{\}]
{\color{outcolor}Out[{\color{outcolor}119}]:} Text(0.5,1,'Filtered Spectrum')
\end{Verbatim}
            
    \begin{center}
    \adjustimage{max size={0.9\linewidth}{0.9\paperheight}}{output_27_1.png}
    \end{center}
    { \hspace*{\fill} \\}
    
    \begin{Verbatim}[commandchars=\\\{\}]
{\color{incolor}In [{\color{incolor}120}]:} \PY{c+c1}{\PYZsh{} Reconstruct the image from the filtered spectrum, keep only the}
          \PY{c+c1}{\PYZsh{} real part for display.}
          \PY{n}{im\PYZus{}new} \PY{o}{=} \PY{n}{fftpack}\PY{o}{.}\PY{n}{ifft2}\PY{p}{(}\PY{n}{im\PYZus{}fft4}\PY{p}{)}\PY{o}{.}\PY{n}{real}
          
          \PY{n}{plt}\PY{o}{.}\PY{n}{figure}\PY{p}{(}\PY{p}{)}
          \PY{n}{plt}\PY{o}{.}\PY{n}{imshow}\PY{p}{(}\PY{n}{im\PYZus{}new}\PY{p}{,} \PY{n}{plt}\PY{o}{.}\PY{n}{cm}\PY{o}{.}\PY{n}{gray}\PY{p}{)}
          \PY{n}{plt}\PY{o}{.}\PY{n}{title}\PY{p}{(}\PY{l+s+s1}{\PYZsq{}}\PY{l+s+s1}{Reconstructed Image}\PY{l+s+s1}{\PYZsq{}}\PY{p}{)}
\end{Verbatim}


\begin{Verbatim}[commandchars=\\\{\}]
{\color{outcolor}Out[{\color{outcolor}120}]:} Text(0.5,1,'Reconstructed Image')
\end{Verbatim}
            
    \begin{center}
    \adjustimage{max size={0.9\linewidth}{0.9\paperheight}}{output_28_1.png}
    \end{center}
    { \hspace*{\fill} \\}
    
    \begin{Verbatim}[commandchars=\\\{\}]
{\color{incolor}In [{\color{incolor}125}]:} \PY{c+c1}{\PYZsh{} Define the fraction of coefficients (in each direction) we keep}
          \PY{n}{keep\PYZus{}fraction} \PY{o}{=} \PY{l+m+mf}{0.5}
          
          \PY{c+c1}{\PYZsh{} Call ff a copy of the original transform. Numpy arrays have a copy}
          \PY{c+c1}{\PYZsh{} method for this purpose.}
          \PY{n}{im\PYZus{}fft5} \PY{o}{=} \PY{n}{im\PYZus{}fft}\PY{o}{.}\PY{n}{copy}\PY{p}{(}\PY{p}{)}
          
          \PY{c+c1}{\PYZsh{} Set r and c to be the number of rows and columns of the array.}
          \PY{n}{r}\PY{p}{,} \PY{n}{c} \PY{o}{=} \PY{n}{im\PYZus{}fft5}\PY{o}{.}\PY{n}{shape}
          
          \PY{c+c1}{\PYZsh{} Set to zero all columns with indices between 0 :r*keep\PYZus{}fraction and}
          \PY{n}{im\PYZus{}fft5}\PY{p}{[}\PY{p}{:}\PY{p}{,} \PY{l+m+mi}{0}\PY{p}{:}\PY{n+nb}{int}\PY{p}{(}\PY{n}{c}\PY{o}{*}\PY{n}{keep\PYZus{}fraction}\PY{p}{)}\PY{p}{]} \PY{o}{=} \PY{l+m+mi}{0}
          
          \PY{n}{plt}\PY{o}{.}\PY{n}{figure}\PY{p}{(}\PY{p}{)}
          \PY{n}{plot\PYZus{}spectrum}\PY{p}{(}\PY{n}{im\PYZus{}fft5}\PY{p}{)}
          \PY{n}{plt}\PY{o}{.}\PY{n}{title}\PY{p}{(}\PY{l+s+s1}{\PYZsq{}}\PY{l+s+s1}{Filtered Spectrum}\PY{l+s+s1}{\PYZsq{}}\PY{p}{)}
\end{Verbatim}


\begin{Verbatim}[commandchars=\\\{\}]
{\color{outcolor}Out[{\color{outcolor}125}]:} Text(0.5,1,'Filtered Spectrum')
\end{Verbatim}
            
    \begin{center}
    \adjustimage{max size={0.9\linewidth}{0.9\paperheight}}{output_29_1.png}
    \end{center}
    { \hspace*{\fill} \\}
    
    \begin{Verbatim}[commandchars=\\\{\}]
{\color{incolor}In [{\color{incolor}126}]:} \PY{c+c1}{\PYZsh{} Reconstruct the image from the filtered spectrum, keep only the}
          \PY{c+c1}{\PYZsh{} real part for display.}
          \PY{n}{im\PYZus{}new} \PY{o}{=} \PY{n}{fftpack}\PY{o}{.}\PY{n}{ifft2}\PY{p}{(}\PY{n}{im\PYZus{}fft5}\PY{p}{)}\PY{o}{.}\PY{n}{real}
          
          \PY{n}{plt}\PY{o}{.}\PY{n}{figure}\PY{p}{(}\PY{p}{)}
          \PY{n}{plt}\PY{o}{.}\PY{n}{imshow}\PY{p}{(}\PY{n}{im\PYZus{}new}\PY{p}{,} \PY{n}{plt}\PY{o}{.}\PY{n}{cm}\PY{o}{.}\PY{n}{gray}\PY{p}{)}
          \PY{n}{plt}\PY{o}{.}\PY{n}{title}\PY{p}{(}\PY{l+s+s1}{\PYZsq{}}\PY{l+s+s1}{Reconstructed Image}\PY{l+s+s1}{\PYZsq{}}\PY{p}{)}
\end{Verbatim}


\begin{Verbatim}[commandchars=\\\{\}]
{\color{outcolor}Out[{\color{outcolor}126}]:} Text(0.5,1,'Reconstructed Image')
\end{Verbatim}
            
    \begin{center}
    \adjustimage{max size={0.9\linewidth}{0.9\paperheight}}{output_30_1.png}
    \end{center}
    { \hspace*{\fill} \\}
    
    \hypertarget{merits-of-svd-vs-fft-image-truncation}{%
\subsection{Merits of SVD vs fft image
truncation}\label{merits-of-svd-vs-fft-image-truncation}}

When the image is truncated with SVD, it takes a high percent of the
original variance for the fingerprint to be recongizable as a
fingerprint. Saving only two PCs saves 73.6\% of the variance, but the
resulting image does not have the structure of a fingerprint. If you
retain 82.4\% of the varaince (10 PCs), then the structure of the
fingerprint starts to show. Saving 20 PCs or 92\% of the variance then
the fingerprint is even more clear. For an intricate image like the
fingerprint, you may need to save a high percentage of the variance of
the original image to retain a tolerable image.

When the image is truncated with fft, you have more control over which
aspect of the image you save. If you specifically keep the corners of
the spectrum then you retain the structure of the fingerprint, but it is
blurry. Keeping half of the spectrum, like in the last two images, you
retain the sharp structure of the fingureprint.

The more you compress your image, the less recognizable the image will
be. You have to decide what aspects of the image you want to keep.


    % Add a bibliography block to the postdoc
    
    
    
    \end{document}
